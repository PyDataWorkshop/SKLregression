\documentclass[SKL-MASTER.tex]{subfiles}
\begin{document}
%========================================================%
% % Working with Linear Models
% % 58
\subsubsection{Theory}
The basic idea of linear regression is to find the set of coefficients of that satisfy \[y = X \beta\] ,
where X is the data matrix. It's unlikely that for the given values of X, we will find a set of
coefficients that exactly satisfy the equation; an error term gets added if there is an inexact
specification or measurement error. Therefore, the equation becomes $y = X \beta + \varepsilon$ , where $\varepsilon$
is assumed to be normally distributed and independent of the X values. Geometrically, this
means that the error terms are perpendicular to X. 

% It's beyond the scope of this book, but it might be worth it to prove E ( X ) = 0 to yourself.

In order to find the set of betas that map the X values to y, we minimize the error term.
This is done by minimizing the residual sum of squares.
This problem can be solved analytically, with the solution being .


\end{document}