\documentclass[SKL-MASTER.tex]{subfiles}
\begin{document}
\section{Using sparsity to regularize models}
\begin{itemize}
\item The least absolute shrinkage and selection operator (LASSO) method is very similar
to ridge regression and LARS. 
\item It's similar to Ridge Regression in the sense that we penalize
our regression by some amount, and it's similar to LARS in that it can be used as a parameter
selection, and it typically leads to a sparse vector of coefficients.
\end{itemize}

\subsubsection{Getting ready}
To be clear, lasso regression is not a panacea. There can be computation consequences to
using lasso regression. As we'll see in this recipe, we'll use a loss function that isn't differential,
and therefore, requires special, and more importantly, performance-impairing workarounds.
\subsection{Implementation}
Let's go back to the trusty \texttt{make\_regression} function and create a dataset with the same
parameters:
\begin{framed}
	\begin{verbatim}
>>> from sklearn.datasets import make_regression
>>> reg_data, reg_target = make_regression(n_samples=200, n_features=500,
n_informative=5, noise=5)
\end{verbatim}
\end{framed}
Next, we need to import the Lasso object:
\begin{framed}
\begin{verbatim}
>>> from sklearn.linear_model import Lasso
>>> lasso = Lasso()
\end{verbatim}
\end{framed}
\subsection{Parameters for LASSO : alpha}
\begin{itemize}
\item Lasso contains many parameters, but the most interesting parameter is alpha. It scales
the penalization term of the Lasso method, which we'll look at later.
\item For now, leave it as 1.
\item As an aside, and much like ridge regression, if this term is 0, lasso is
equivalent to linear regression:
\end{itemize}

\begin{framed}
	\begin{verbatim}
>>> lasso.fit(reg_data, reg_target)
\end{verbatim}
\end{framed}
Again, let's see how many of the coefficients remain nonzero:
\begin{framed}
\begin{verbatim}
>>> np.sum(lasso.coef_ != 0)
9
>>> lasso_0 = Lasso(0)
>>> lasso_0.fit(reg_data, reg_target)
>>> np.sum(lasso_0.coef_ != 0)
500
\end{verbatim}
\end{framed}
%========================================================%
\begin{itemize}
\item None of our coefficients turn out to be 0, which is what we expect. 
\item Actually, if you run this, you
might get a warning from scikit-learn that advises you to choose LinearRegression.
\end{itemize}


%How it works...
%For linear regression, we minimized the squared error. Here, we're still going to minimize the
%squared error, but we'll add a penalization term that will induce the scarcity. The equation
%looks like the following:
%i 1 e + 
%An alternate way of looking at this is to minimize the residual sum of squares:
%( ) 1 RSS  such that  < 
%This constraint is what leads to the scarcity. Lasso regression's constraint creates a hypercube
%around the origin (the coefficients being the axis), which means that the most extreme points
%are the corners, where many of the coefficients are 0. Ridge regression creates a hypersphere
%due to the constraint of the l2 norm being less than some constant, but it's very likely that
%coefficients will not be zero even if they are constrained.
%================================================= %
\newpage
\subsection{Lasso cross-validation}
Choosing the most appropriate lambda is a critical problem. We can specify the lambda
ourselves or use cross-validation to find the best choice given the data at hand:
\begin{framed}
\begin{verbatim}
>>> from sklearn.linear_model import LassoCV
>>> lassocv = LassoCV()
>>> lassocv.fit(reg_data, reg_target)
\end{verbatim}
\end{framed}
\begin{itemize}
\item \texttt{lassocv} will have, as an attribute, the most appropriate lambda. 
\item \textbf{Remark:} scikit-learn mostly uses
alpha in its notation, but the literature uses lambda:
\end{itemize}
\begin{framed}
\begin{verbatim}
>>> lassocv.alpha_
0.80722126078646139
\end{verbatim}
\end{framed}
The number of coefficients can be accessed in the regular manner:
\begin{framed}
	\begin{verbatim}
>>> lassocv.coef_[:5]
array([0., 42.41, 0.,0., -0.])
\end{verbatim}
\end{framed}
%========================================================%
%Working with Linear Models
%72
Letting \texttt{lassocv} choose the appropriate best fit leaves us with 11 nonzero coefficients:

\begin{framed}
\begin{verbatim}
>>> np.sum(lassocv.coef_ != 0)
11
\end{verbatim}
\end{framed}


\subsubsection{Lasso for feature selection}
\begin{itemize}
\item Lasso can often be used for feature selection for other methods. 
\item For example, you might run
lasso regression to get the appropriate number of features, and then use these features in
another algorithm.
\item To get the features we want, create a masking array based on the columns that aren't zero,
and then filter to keep the features we want:
\end{itemize}

\begin{framed}
	\begin{verbatim}
	>>> mask = lassocv.coef_ != 0
	>>> new_reg_data = reg_data[:, mask]
	>>> new_reg_data.shape
	(200, 11)
	\end{verbatim}
\end{framed}

\end{document}